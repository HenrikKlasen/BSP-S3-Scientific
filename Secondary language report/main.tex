\documentclass[conference,compsoc]{IEEEtran}
\usepackage{datetime}
\usepackage{caption}
\usepackage{listings}
\usepackage{algorithm} 
\usepackage{algpseudocode} 
\lstdefinestyle{mystyle}{
    basicstyle=\small\sffamily,
	numbers=left,
	numberstyle=\tiny,
	frame=tb,
	breakatwhitespace=true,  
    	breaklines=true,
	columns=fullflexible,
	showstringspaces=false}
\usepackage[utf8]{inputenc}
\usepackage{menukeys}

% *** CITATION PACKAGES ***
%
\ifCLASSOPTIONcompsoc
  % IEEE Computer Society needs nocompress option
  % requires cite.sty v4.0 or later (November 2003)
  \usepackage[nocompress]{cite}
\else  
  % normal IEEE
  \usepackage{cite} 
\fi 
% cite.sty was written by Donald Arseneau
% V1.6 and later of IEEEtran pre-defines the format of the cite.sty package
% \cite{} output to follow that of the IEEE. Loading the cite package will
% result in citation numbers being automatically sorted and properly
% "compressed/ranged". e.g., [1], [9], [2], [7], [5], [6] without using
% cite.sty will become [1], [2], [5]--[7], [9] using cite.sty. cite.sty's
% \cite will automatically add leading space, if needed. Use cite.sty's
% noadjust option (cite.sty V3.8 and later) if you want to turn this off
% such as if a citation ever needs to be enclosed in parenthesis.
% cite.sty is already installed on most LaTeX systems. Be sure and use
% version 5.0 (2009-03-20) and later if using hyperref.sty.
% The latest version can be obtained at:
% http://www.ctan.org/pkg/cite
% The documentation is contained in the cite.sty file itself.
%
% Note that some packages require special options to format as the Computer
% Society requires. In particular, Computer Society  papers do not use
% compressed citation ranges as is done in typical IEEE papers
% (e.g., [1]-[4]). Instead, they list every citation separately in order
% (e.g., [1], [2], [3], [4]). To get the latter we need to load the cite
% package with the nocompress option which is supported by cite.sty v4.0
% and later.

% *** GRAPHICS RELATED PACKAGES ***
%
\ifCLASSINFOpdf
  % \usepackage[pdftex]{graphicx}
  % declare the path(s) where your graphic files are
  % \graphicspath{{../pdf/}{../jpeg/}}
  % and their extensions so you won't have to specify these with
  % every instance of \includegraphics
  % \DeclareGraphicsExtensions{.pdf,.jpeg,.png}
\else
  % or other class option (dvipsone, dvipdf, if not using dvips). graphicx
  % will default to the driver specified in the system graphics.cfg if no
  % driver is specified.
  % \usepackage[dvips]{graphicx}
  % declare the path(s) where your graphic files are
  % \graphicspath{{../eps/}}
  % and their extensions so you won't have to specify these with
  % every instance of \includegraphics
  % \DeclareGraphicsExtensions{.eps}
\fi
% graphicx was written by David Carlisle and Sebastian Rahtz. It is
% required if you want graphics, photos, etc. graphicx.sty is already
% installed on most LaTeX systems. The latest version and documentation
% can be obtained at: 
% http://www.ctan.org/pkg/graphicx
% Another good source of documentation is "Using Imported Graphics in
% LaTeX2e" by Keith Reckdahl which can be found at:
% http://www.ctan.org/pkg/epslatex
%
% latex, and pdflatex in dvi mode, support graphics in encapsulated
% postscript (.eps) format. pdflatex in pdf mode supports graphics
% in .pdf, .jpeg, .png and .mps (metapost) formats. Users should ensure
% that all non-photo figures use a vector format (.eps, .pdf, .mps) and
% not a bitmapped formats (.jpeg, .png). The IEEE frowns on bitmapped formats
% which can result in "jaggedy"/blurry rendering of lines and letters as
% well as large increases in file sizes.
%
% You can find documentation about the pdfTeX application at:
% http://www.tug.org/applications/pdftex


% *** MATH PACKAGES ***
%
\usepackage{amsmath}
% A popular package from the American Mathematical Society that provides
% many useful and powerful commands for dealing with mathematics.
%
% Note that the amsmath package sets \interdisplaylinepenalty to 10000
% thus preventing page breaks from occurring within multiline equations. Use:
%\interdisplaylinepenalty=2500
% after loading amsmath to restore such page breaks as IEEEtran.cls normally
% does. amsmath.sty is already installed on most LaTeX systems. The latest
% version and documentation can be obtained at:
% http://www.ctan.org/pkg/amsmath

% *** SPECIALIZED LIST PACKAGES ***
%
%\usepackage{algorithmic}
% algorithmic.sty was written by Peter Williams and Rogerio Brito.
% This package provides an algorithmic environment fo describing algorithms.
% You can use the algorithmic environment in-text or within a figure
% environment to provide for a floating algorithm. Do NOT use the algorithm
% floating environment provided by algorithm.sty (by the same authors) or
% algorithm2e.sty (by Christophe Fiorio) as the IEEE does not use dedicated
% algorithm float types and packages that provide these will not provide
% correct IEEE style captions. The latest version and documentation of
% algorithmic.sty can be obtained at:
% http://www.ctan.org/pkg/algorithms
% Also of interest may be the (relatively newer and more customizable)
% algorithmicx.sty package by Szasz Janos:
% http://www.ctan.org/pkg/algorithmicx


% *** ALIGNMENT PACKAGES ***
%
%\usepackage{array}
% Frank Mittelbach's and David Carlisle's array.sty patches and improves
% the standard LaTeX2e array and tabular environments to provide better
% appearance and additional user controls. As the default LaTeX2e table
% generation code is lacking to the point of almost being broken with
% respect to the quality of the end results, all users are strongly
% advised to use an enhanced (at the very least that provided by array.sty)
% set of table tools. array.sty is already installed on most systems. The
% latest version and documentation can be obtained at:
% http://www.ctan.org/pkg/array

% IEEEtran contains the IEEEeqnarray family of commands that can be used to
% generate multiline equations as well as matrices, tables, etc., of high
% quality.

% *** SUBFIGURE PACKAGES ***
%\ifCLASSOPTIONcompsoc
%  \usepackage[caption=false,font=footnotesize,labelfont=sf,textfont=sf]{subfig}
%\else
%  \usepackage[caption=false,font=footnotesize]{subfig}
%\fi
% subfig.sty, written by Steven Douglas Cochran, is the modern replacement
% for subfigure.sty, the latter of which is no longer maintained and is
% incompatible with some LaTeX packages including fixltx2e. However,
% subfig.sty requires and automatically loads Axel Sommerfeldt's caption.sty
% which will override IEEEtran.cls' handling of captions and this will result
% in non-IEEE style figure/table captions. To prevent this problem, be sure
% and invoke subfig.sty's "caption=false" package option (available since
% subfig.sty version 1.3, 2005/06/28) as this is will preserve IEEEtran.cls
% handling of captions.
% Note that the Computer Society format requires a sans serif font rather
% than the serif font used in traditional IEEE formatting and thus the need
% to invoke different subfig.sty package options depending on whether
% compsoc mode has been enabled.
%
% The latest version and documentation of subfig.sty can be obtained at:
% http://www.ctan.org/pkg/subfig

% *** FLOAT PACKAGES ***
%
%\usepackage{fixltx2e}
% fixltx2e, the successor to the earlier fix2col.sty, was written by
% Frank Mittelbach and David Carlisle. This package corrects a few problems
% in the LaTeX2e kernel, the most notable of which is that in current
% LaTeX2e releases, the ordering of single and double column floats is not
% guaranteed to be preserved. Thus, an unpatched LaTeX2e can allow a
% single column figure to be placed prior to an earlier double column
% figure.
% Be aware that LaTeX2e kernels dated 2015 and later have fixltx2e.sty's
% corrections already built into the system in which case a warning will
% be issued if an attempt is made to load fixltx2e.sty as it is no longer
% needed.
% The latest version and documentation can be found at:
% http://www.ctan.org/pkg/fixltx2e

%\usepackage{stfloats}
% stfloats.sty was written by Sigitas Tolusis. This package gives LaTeX2e
% the ability to do double column floats at the bottom of the page as well
% as the top. (e.g., "\begin{figure*}[!b]" is not normally possible in
% LaTeX2e). It also provides a command:
%\fnbelowfloat
% to enable the placement of footnotes below bottom floats (the standard
% LaTeX2e kernel puts them above bottom floats). This is an invasive package
% which rewrites many portions of the LaTeX2e float routines. It may not work
% with other packages that modify the LaTeX2e float routines. The latest
% version and documentation can be obtained at:
% http://www.ctan.org/pkg/stfloats
% Do not use the stfloats baselinefloat ability as the IEEE does not allow
% \baselineskip to stretch. Authors submitting work to the IEEE should note
% that the IEEE rarely uses double column equations and that authors should try
% to avoid such use. Do not be tempted to use the cuted.sty or midfloat.sty
% packages (also by Sigitas Tolusis) as the IEEE does not format its papers in
% such ways.
% Do not attempt to use stfloats with fixltx2e as they are incompatible.
% Instead, use Morten Hogholm'a dblfloatfix which combines the features
% of both fixltx2e and stfloats:
%
% \usepackage{dblfloatfix}
% The latest version can be found at:
% http://www.ctan.org/pkg/dblfloatfix

% *** PDF, URL AND HYPERLINK PACKAGES ***
%
%\usepackage{url}
% url.sty was written by Donald Arseneau. It provides better support for
% handling and breaking URLs. url.sty is already installed on most LaTeX
% systems. The latest version and documentation can be obtained at:
% http://www.ctan.org/pkg/url
% Basically, \url{my_url_here}.

% *** Do not adjust lengths that control margins, column widths, etc. ***
% *** Do not use packages that alter fonts (such as pslatex).         ***
% There should be no need to do such things with IEEEtran.cls V1.6 and later.
% (Unless specifically asked to do so by the journal or conference you plan
% to submit to, of course. )

% correct bad hyphenation here
\hyphenation{op-tical net-works semi-conduc-tor}
   
\usepackage{hyperref}
 
\begin{document}
\lstset{style=mystyle}
% 
% paper title
% Titles are generally capitalized except for words such as a, an, and, as,
% at, but, by, for, in, nor, of, on, or, the, to and up, which are usually
% not capitalized unless they are the first or last word of the title.
% Linebreaks \\ can be used within to get better formatting as desired.
% Do not put math or special symbols in the title.
\title{
	OrbitCalc - Eine vereinfachte Version des Sonnensystems\\
{\small \today~-~\currenttime}}

 
% author names and affiliations
% use a multiple column layout for up to three different
% affiliations
\author{\IEEEauthorblockN{Henrik Klasen}
\IEEEauthorblockA{University of Luxembourg\\
Email: henrik.klasen.001@student.uni.lu}
\\
{\bf Dieses Projekt wurde durchgeführt unter der Aufsicht von:}\\
\IEEEauthorblockN{Gabriel Garcia}
\IEEEauthorblockA{University of Luxembourg\\
Email: gabriel.garcia@uni.lu}%
}

% conference papers do not typically use \thanks and this command
% is locked out in conference mode. If really needed, such as for
% the acknowledgment of grants, issue a \IEEEoverridecommandlockouts
% after \documentclass

% for over three affiliations, or if they all won't fit within the width
% of the page (and note that there is less available width in this regard for
% compsoc conferences compared to traditional conferences), use this
% alternative format:
% 
%\author{\IEEEauthorblockN{Michael Shell\IEEEauthorrefmark{1},
%Homer Simpson\IEEEauthorrefmark{2},
%James Kirk\IEEEauthorrefmark{3}, 
%Montgomery Scott\IEEEauthorrefmark{3} and
%Eldon Tyrell\IEEEauthorrefmark{4}}
%\IEEEauthorblockA{\IEEEauthorrefmark{1}School of Electrical and Computer Engineering\\
%Georgia Institute of Technology,
%Atlanta, Georgia 30332--0250\\ Email: see http://www.michaelshell.org/contact.html}
%\IEEEauthorblockA{\IEEEauthorrefmark{2}Twentieth Century Fox, Springfield, USA\\
%Email: homer@thesimpsons.com}
%\IEEEauthorblockA{\IEEEauthorrefmark{3}Starfleet Academy, San Francisco, California 96678-2391\\
%Telephone: (800) 555--1212, Fax: (888) 555--1212}
%\IEEEauthorblockA{\IEEEauthorrefmark{4}Tyrell Inc., 123 Replicant Street, Los Angeles, California 90210--4321}}




% use for special paper notices
%\IEEEspecialpapernotice{(Invited Paper)}




% make the title area
\maketitle

%to remove for your report
%\footnote{}

% As a general rule, do not put math, special symbols or citations
% in the abstract
\begin{abstract}
Dieses Bachelor Semester Projekt behandelt den Prozess der mathematischen Modellierung für wissenschaftliche Simulationen. Als Fallbeispiel hierfür wird eine dreidimensionale Simulation für das Sonnensystem aufgeführt. (260/800 $\rightarrow$ ca. 26\%)
\end{abstract}

% no keywords

% For peer review papers, you can put extra information on the cover
% page as needed:
% \ifCLASSOPTIONpeerreview
% \begin{center} \bfseries EDICS Category: 3-BBND \end{center}
% \fi
%
% For peerreview papers, this IEEEtran command inserts a page break and
% creates the second title. It will be ignored for other modes.
\IEEEpeerreviewmaketitle



\section{Wissenschaftliche Arbeit (41/400)} 
In der wissenschaftlichen Arbeit dieses Projektes geht es zunächst um das Finden einer Schritt für Schritt Anleitung zum Erstellen eines mathematischen Modells für eine beliebige Umgebung. Hier wird die Quelle \cite{ComputationalModelsIntroduction} hauptsächlich genutzt, um die verschiedenen Aspekte des Modellierens zu erarbeiten.\\
So ist der erste Schritt die Observierung der Umgebung. Hier kann man sich bei der Erstellung des Modells die Fragen stellen: Welches Verhalten weist die Modellvorlage auf? Welche Randfälle gibt es? \\
Im zweiten Schritt geht es um die Problembeschreibung. Hierbei geht es vor allem darum, dass das zu simulierende Verhalten aus der Modellvorlage exakt beschrieben wird und man auch erste Vereinfachungen des Modells stellt. Es hilft hier auch eine Liste an Funktionen aufzusetzen, die in dem Modell sein \emph{müssen}, und eine Liste mit Funktionen, die das Modell erweitern würden, jedoch nicht zum ursprünglichen Zweck des Modells nötig sind. \\
Der dritte Schritt in dieser Ausarbeitung dient der mathematischen Modellierung. Hierbei wird das beobachtete Verhalten mit mathematischen Formeln beschrieben, oder wenn nicht anders möglich angenähert. Es ist hierbei von Vorteil, wenn hier weitere Vereinfachungen eingeführt werde, deren Auswirkungen auf das Modell marginal sind. \\ 
Mit dem vierten Schritt wird das algorithmische Design angeführt. In diesem geht es darum, die Ergebnisse aus der mathematischen Modellierung in (ressourcenschonende) Algorithmen umzuwandeln. Vor allem rekursive Formeln sind hierbei in iterative Prozeduren umzuwandeln, da Rekursion sehr rechenintensiv sein kann, oder aber Variablen einzuführen, die das Zwischenergebnis und somit für spätere Berechnungen verfügbar machen. 
Abschließend gibt es einige Leitfragen um die Schritte zu vereinfachen. 
\begin{enumerate}
	\item Welches Verhalten ist zu beobachten?
	\item Was wollen wir simulieren? Was für ein Problem ist zu simulieren? Welche Eigenschaften soll das Modell haben?
	\item Welche Formeln beschreiben die zu simulierende Umgebung und diegenannten Eigenschaften? Gibt es ausreichende Annäherungen um Ressourcen zu sparen?
	\item Wie können wir die Formeln in Algorithmen umwandeln?
\end{enumerate}
In den danach folgenden Sektionen geht es vor allem um das Fallbeispiel und die Durchführung der Schritte 1-3. \\
Noch im wissenschaftlichen Teil geht es um Beobachtungen der Modellvorlage. So gibt es die Beobachtung, dass die Sonne im Zentrum des Modells ist, und sich nur minimal (quasi vernachlässigbar) bewegt. Somit wird die Sonne im Modell zur Vereinfachung als statisch angenommen. 
\section{Technische Arbeit}
\begin{thebibliography}{1}
\bibitem[BiCS(2021)]{bics-bsp-report-template}
\newblock {BiCS Bachelor Semester Project Report Template}.
\newblock {https://github.com/nicolasguelfi/lu.uni.course.bics.global}
\newblock {University of Luxembourg, BiCS - Bachelor in Computer Science (2021).}

\bibitem[BiCS(2021)] {bics-bsp-reference-document}
{Bachelor in Computer Science}:
\newblock {BiCS Semester Projects Reference Document}.
\newblock Technical report, University of Luxembourg (2021)

\bibitem[Armstrong and Green(2017)]{armstrong2017guidelinesforscience}
J~Scott Armstrong and Kesten~C Green.
\newblock Guidelines for science: Evidence and checklists.
\newblock \emph{Scholarly Commons}, pages 1--24, 2017.
\newblock {https://repository.upenn.edu/marketing\_papers/181/}

	\bibitem[BiCS(2021)]{bics-bsp-report-template}
	\newblock {BiCS Bachelor Semester Project Report Template}.
	\newblock {https://github.com/nicolasguelfi/lu.uni.course.bics.global}
	\newblock {University of Luxembourg, BiCS - Bachelor in Computer Science (2021).}
	
	\bibitem[BiCS(2021)] {bics-bsp-reference-document}
	{Bachelor in Computer Science}:
	\newblock {BiCS Semester Projects Reference Document}.
	\newblock Technical report, University of Luxembourg (2021)
	
	\bibitem[Armstrong and Green(2017)]{armstrong2017guidelinesforscience}
	J~Scott Armstrong and Kesten~C Green.
	\newblock Guidelines for science: Evidence and checklists.
	\newblock \emph{Scholarly Commons}, pages 1--24, 2017.
	\newblock {https://repository.upenn.edu/marketing\_papers/181/}
	
	\bibitem[KaufmannFranzinMenegaisPozzer]{PhysicsSimulationsLargeWorlds}
	Lorenzo Schwertner Kaufmann, Flavio Paulus Franzin, Roberto Menegais, Cesar Tadeu Pozzer.
	\newblock Accurate Real-Time Physics Simulation for Large Worlds.
	\newblock \emph{Universidade Federal de Santa Maria, Santa Maria, Brazil}, 2021.
	
	\bibitem[MurinKompisKutis]{ComputationalModelling}
	Justin Murin, Vladimir Kompis, Vladimir Kutis.
	\newblock Computational Modelling and Advanced Simulations. \\
	\newblock \emph{Springer}, 978-94-007-0317-9-1, 2011.
	
	\bibitem[HeisterRebholz]{ScientificComputing}
	Timo Heister, Leo G. Rebholz.
	\newblock Scientific Computing for Scientists and Engineers.
	\newblock \emph{De Gruyter}, 2023.
	
	\bibitem[Garrido]{ComputationalModelsIntroduction}
	Jose M. Garrido
	\newblock Introduction to Computational Models with Python.
	\newblock \emph{Chapman and Hall/CRC}, 2016.
	
	\bibitem[Unity]{UnityDoc}
	\newblock \href{https://docs.unity3d.com/Manual/index.html}{https://docs.unity3d.com/Manual/index.html}

\end{thebibliography}
% that's all folks

\end{document}


